\section*{Introducción}
El problema del agente viajero o TSP (Traveling Salesman Problem) es quizá el problema de optimización combinatoria más conocido. El problema consiste en N puntos o ciudades a visitar, se tiene un costo para el viaje entre cada par de puntos. Se asume que un vendedor, comenzando de cualquiera de las ciudades, tiene que visitar todos los puntos exactamente una vez y regresar al punto inicial. El objetivo es encontrar un recorrido óptimo para el que la distancia total o costo total del viaje se minimiza.
\par En este trabajo se trata con un caso especial de este problema y se intenta optimizar los recorridos mediante el uso de métodos de búsqueda local y el uso de un algoritmo genético con distintas consideraciones en cuanto a representación y operador de cruza. También se combinan los algoritmos utilizados para intentar brindar mejores soluciones al problema que las que se obtienen mediante el uso de estas técnicas por separado.
